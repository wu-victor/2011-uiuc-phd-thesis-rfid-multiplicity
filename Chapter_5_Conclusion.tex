\chapter{CONCLUSION}
\label{Section: Conclusion}

\section{Summary of Work}
Passive RFID enables the design and implementation of continuous pervasive spaces. We argue that tag multiplicity allows us to design and deploy robust, distributed physical information systems, achieving this pervasivity. In Chapter \ref{Section: Introduction}, we explain continuous pervasive spaces, including several salient characteristics. We also explain tag-based information systems as the unique enabler of these spaces. We discuss applications and related work, as well as supporting and related technologies.

In Chapter \ref{Section: Tracking Protocols}, we study tracking protocols in tag-based information systems. We investigate two scenarios. In the first, we consider space-time correlations, which are functions stored inline in tags, indicating parts of a digital trail stored by a previous user. Tags are assumed to have infinite storage space. In the second scenario, we consider forest search and rescue. Tags are deployed densely in space, and digital trails are again stored. However, we focus on replacement algorithms for the data stored in tags, since we assume tags have finite storage space, and there are many digital trails from many users.

In Chapter \ref{Section: Storage Access Protocols}, we study storage systems in tag-based information systems. In particular, we consider two scenarios. In the first, we assume a dense tag deployment. An interrogator therefore scans many tags for a given scan range. So we consider efficient storage access protocols to read from and write to tags. We achieve this efficiency using constrained access protocols. In the second scenario, we assume a more moderate tag deployment. In this case, we consider more traditional random access protocols. We also study tag storage lifetime and tag granularity.

In Chapter \ref{Section: Distributed Passive RFID Computing}, we study computing in a tag-based information system. We consider several computing-based metrics and evaluate them in a simulation study.

\section{Continuous Pervasivity}
Passive RFID tag multiplicity allows us to design and deploy robust, distributed physical information systems for continuous pervasive spaces. Here, we aggregate many of these pervasive characteristics discussed in other sections. In particular, we frame the ideas within the characteristics outlined in Section \ref{Section: Introduction: Pervasive Computing and Continuous Pervasive Spaces: Characteristics of Continuous Pervasive Spaces}.

\subsection{Service Richness and Elasticity}
Tag multiplicity is a powerful concept, forming the foundation of service richness and elasticity. In particular, a dense deployment of tags creates a continuous pervasivity. That is, a user experiences space-time continuously in the services she uses, creating a very rich environment. We see that these services can be framed in a general manner by labeling them as ``computing.'' In Section \ref{Section: Distributed Passive Computing: Local Computations}, we study local computing, where an interrogator scans a set of tags and interacts with their stored instructions and data. This is extended to distributed computing in a larger system in Section \ref{Section: Distributed Passive RFID Computing: System Model}, where multiple interrogators move around in a large physical area, scanning tags and operating on them. We see that distributed computing-based metrics depend on interrogator dynamics. So as we use more resources (such as increasing the tag storage density and scan range of interrogators), these metrics improve (such as system connectivity), and in turn, the services become more powerful (such as communications). That is, service availability and integrity are very elastic. Marginal increases in resources translate quickly to marginal increases in service performance. 

We also study more specific services for continuous pervasive spaces. In Section \ref{Section: Tracking Protocols: Space-time Correlations}, the digital trails stored for tracking can be fat or dense. This is customizable based on changes in tag deployment or algorithmic adjustments. The flexibility provides a rich range of services, such as a user backtracking, trail following, or search and rescue. For the specific case of forest search and rescue in Section \ref{Section: Tracking Protocols: Forest Search and Rescue}, a user may choose to maximize her own possibility of successful rescue, or instead sacrifice some of this for other users. This flexibility is achieved through choosing different replacement algorithms, and is ultimately made possible through tag multiplicity.

In Chapter \ref{Section: Storage Access Protocols}, we study storage systems in tag-based information systems. In particular, we consider two scenarios. In the first, we assume a dense tag deployment. An interrogator therefore scans many tags for a given scan range. So we consider efficient storage access protocols to read from and write to tags. We achieve this efficiency using constrained access protocols. In the second scenario, we assume a more moderate tag deployment. In this case, we consider more traditional random access protocols. We also study tag storage lifetime and tag granularity. Storage naturally has a wide range of application domains, and is indispensable to many of those. There are many different needs and goals for these services. We support this richness by considering different storage access protocols for tag-based information systems. Furthermore, different protocols can even be implemented simultaneously. That is, sections of a tag's storage can be assigned for different purposes. Or we can even write to tags using one protocol, and then read from those same tags using another. This may depend on the current wireless environment or interrogator resources. Storage access protocols are thus elastic in their resulting availability, depending on the system conditions.

\subsection{Minimum User Distraction}
A defining characteristic of pervasive computing is that technology should fade into the background, as perceived by a user. That is, a user should be minimally distracted by the technology itself, and instead focus her cognitive efforts on the task at hand, aided by the service provided by the system. This is achieved in the case of continuous pervasive spaces through tag multiplicity. Since tags are distributed densely in space-time, the resulting services feel continuous to the user in space-time. That is, there is no gap, and as a result, the awareness of the supporting technology eventually fades away.

One tradeoff in this respect is functionality versus minimum cognitive load. If more features or customizations are made available to the user, it requires more attention on her part. Conversely, if the system itself makes assumptions and adapts accordingly, the user is released from much of that cognitive load. Clearly, this depends very much on the service itself. Our systems can operate at many points on this spectrum, achieving maximum service richness, for the maximum tolerable cognitive complexity. In Section \ref{Section: Tracking Protocols: Space-time Correlations}, there are many parameters and levels of complexity in the various tracking algorithms. In the space-time correlation functions, we can allow the user to specify the type of weighting function and direction. To hide some complexity, we can allow the user to only choose between fat or dense arrow fields. To further simplify, we can allow the user to select between backtracking, trail following, or search and rescue modes. The interrogator automatically makes the algorithmic and parameter adjustments accordingly. Similarly for forest search and rescue in Section \ref{Section: Tracking Protocols: Forest Search and Rescue}, the algorithms can be tailored to maximize the personal benefit of being found, or the system benefit of other users being found. Again, we can trade off user-defined functionality with minimum cognitive load.

In Chapter \ref{Section: Storage Access Protocols}, minimum user distraction is again possible because of tag multiplicity. If tags are few and far between, users are burdened with interacting with a small set of tags, or even individual tags for storage purposes. Instead, our system allows a user to store in space-time localities. She does not even need to be aware of the presence of tags. Rather, the storage medium is purposely abstract. That is, rather than storing information in \emph{any set of tags}, she stores information \emph{anywhere} in the continuous pervasive space. Furthermore, there are storage features in addition to space-time coordinates. Section \ref{Section: Storage Access Protocols: Random Access Protocols: Strategies for Privacy and Redundancy} discusses many ways to protect privacy. Traditional password-based mechanisms place a heavy burden on users. We discuss distributed approaches that work automatically, lifting the cognitive burden from the user. These distributed approaches are again based on tag-multiplicity.

In Chapter \ref{Section: Distributed Passive RFID Computing}, similar to storage, we show that computing starts in space-time localities, rather than at a per tag level. Though we can adjust tag storage density and tag deployment granularity, the user should not be made aware. With a sufficient tag density, we can achieve significant computing speeds for various applications in a single interrogator scan. Furthermore, as multiple interrogators traverse a space, the local computations become distributed computations. And again, users are oblivious to this, to their benefit. We also study computation time utilization. Interrogators can learn the statistics of user dynamics, automatically adjusting the computation time in order to maximize utilization. Users are not distracted, and not even made aware of the supporting computations.

\subsection{System Distributedness and Scalability}
Tag multiplicity is foundational to system distributedness and scalability in continuous pervasive spaces. In particular, distributedness of tags provides a high level of fault-tolerance. There is no single point of failure. The system is essentially uniformly spread out in terms of infrastructure. If any single part fails, such as a small locality of tags, the system still functions. In fact, we expect tags to fail regularly. They have finite lifetimes. But since they each fail independently, and we can assume their remaining lifetimes to be memoryless, we do not need to keep track of individual tags. (In the case of specific disasters causing premature tag failures, a system administrator can easily locate the disaster and affected tags.) Therefore, we only need to periodically deploy new tags, uniformly in the space. Interrogators moving through the space can record tag storage density in different areas. But in general, we do not need a detailed health report to maintain the system. We only need to deploy new tags. Similarly, expansion is simple, leading to a highly scalable design. In particular, our continuous pervasive space is scalable in physical size, tag storage density, and tag granularity. Again, to achieve these goals, we merely deploy additional tags, while requiring minimal information to do so.

\subsection{Energy-Efficiency and Cost-Efficiency}
Passive RFID is inherently energy-efficient. That is, a passive tag has no on-board battery. It is powered by the interrogator scanning it. Therefore, energy (from any source) is only expended when it is being used. From that perspective, there is 100\% power efficiency. No energy is being wasted. Furthermore, in Section \ref{Section: Distributed Passive Computing: Local Computations}, we show the energy consumption of tag scanning, and in particular, singulation. We demonstrate that as we scan more tags, the energy cost of each tag scan increases. Consequently, if we desire higher tag storage density, we should use fewer tags, but each with more storage capacity, in order to minimize energy consumption. However, as we see in Section \ref{Section: Storage Access Protocols: Tag Storage Lifetime and Tag Granularity}, to increase tag granularity, we need to deploy more tags. This gives us a smaller distance between tags, but of course the energy consumption for singulation increases as a tradeoff. In any case, with the elasticity of tag deployment, a system administrator can finely tune her system to achieve the desired goals.

Cost-efficiency follows from energy efficiency, at least for the cost of energy. However, other costs include system maintenance. As already mentioned, system maintenance is remarkably simple. There is no need to tightly observe the system, monitoring which specific tags have failed. Rather, knowing the average failure rates (lifetimes) of tags is often already sufficient. Furthermore, interrogators can also report to the system administrator about the general health of tags deployed in the space. Therefore, the main cost (in addition to energy consumption of interrogators) is tag deployment. Since individual tags are very inexpensive, we can estimate how many tags are necessary, and therefore accurately deploy the right quantity (for an initial deployment or a maintenance deployment) to tightly match the demand with the supply. Therefore, there is very little waste.

\section{Final Remarks}
In this work, we investigate how tag multiplicity is the key enabler for continuous pervasive spaces. In particular we can design and deploy robust, distributed, physical information systems. Right now, passive RFID technology is becoming a mature business technology oriented for supply chain and warehouse management systems. We are hopeful that the ideas here help propel RFID beyond identification, toward more exciting application domains, such as pervasive computing. There is much work yet to be done.